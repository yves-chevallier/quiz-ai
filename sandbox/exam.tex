\documentclass[french,a4paper,addpoints,11pt]{exam}
\usepackage[T1]{fontenc}
\usepackage[french]{babel}
\usepackage{fontspec}
\usepackage{hexam}
\usepackage{bashful}
\usepackage{booktabs}

\setlength{\itemsep}{0pt}
\setlength{\parskip}{0pt}
\setlength{\topsep}{0pt}

\lstMakeShortInline|
\pointsdroppedatright
\renewcommand{\checkboxeshook}{%
  \settowidth{\leftmargin}{W.}
  \setlength{\topsep}{1em}
  \labelwidth\leftmargin\advance\labelwidth-\labelsep%
}
\renewcommand{\partshook}{\setlength{\itemsep}{1em}}

\setlength{\gridsize}{5mm}
\setlength{\gridlinewidth}{0.1pt}
\colorgrids
\definecolor{GridColor}{gray}{0.5}
\definecolor{FillWithDottedLinesColor}{gray}{0.5}

\thispagestyle{empty}

\title{Quiz Python\\ QZ-01}

\author{Prof. Yves Chevallier}
\department{Département TIN}
\school{Haute École d'Ingénierie et de Gestion du Canton de Vaud}

\date{8 octobre 2025}

\begin{document}

\begin{coverpages}
    \maketitle
    \thispagestyle{empty}

    % \begin{center}
    %     \vskip 2em
    %     \gradetable
    % \end{center}

    \vspace{0.1in}

    \fbox{\fbox{\parbox{5.5in}{
                \textbf{Consignes :}
                \vskip 1em%
                \begin{itemize}
                    \item Écrire votre nom et votre prénom sur la première page.
                    \item Écrire lisiblement, au stylo ou au crayon à papier gras.
                    \item Répondre aux questions dans les espaces appropriés.
                    \item Relire toutes vos réponses avant de rendre votre travail.
                    \item Rendre toutes les feuilles de ce quiz.
                    \item Aucun moyen de communication autorisé.
                    \item Toutes les réponses concernent le langage Python 3.10.
                \end{itemize}
            }}}

\end{coverpages}
\setcounter{page}{2}
\begin{questions}


    \question

    Python est un langage qui a été créé par Guido van Rossum en 1991. Guido van Rossum est souvent appelé le BDFL de Python. Que signifie BDFL ?

    \begin{multicols}{2}
        \begin{checkboxes}
            \choice Big Data Friendly Language
            \choice Best Data Format Language
            \CorrectChoice Benevolent Dictator For Life
            \choice Bold Decision for Life
            \choice Basic Data Flow Language
            \choice Built-in Data Function Library
        \end{checkboxes}
    \end{multicols}

    \question

    Que signifie REPL en Python ?

    \begin{checkboxes}
        \choice Runtime Environment for Python Libraries, utilisé pour exécuter des modules Python
        \choice Resource Evaluation Processing Loop, principalement utilisé dans les systèmes embarqués
        \CorrectChoice Read-Eval-Print Loop, utile pour tester rapidement des instructions Python
        \choice Repetitive Execution Program Library, déconseillé pour les débutants
        \choice Real-time Execution Programming Language, utilisé pour les applications temps réel
        \choice Built-in Data Function Library
    \end{checkboxes}

    \question

    Quel est le programme utilisé préférablement pour utiliser Python comme calculatrice en ligne de commande ?

    \begin{multicols}{3}
        \begin{checkboxes}
            \choice calc
            \choice python
            \CorrectChoice ipython
            \choice qpython
            \choice jupyter
            \choice notebook
        \end{checkboxes}
    \end{multicols}

    \question

    Inscrire après chaque syntaxe Python le type correspondant à l'expression.

    \begin{multicols}{2}
        \begin{enumerate}
            \item \lstinline|[1]| \hspace{1em} \fillin[liste]
            \item \lstinline|{1}| \hspace{1em} \fillin[set]
            \item \lstinline|{1:2}| \hspace{1em} \fillin[dictionnaire]
            \item \lstinline|(1,2)| \hspace{1em} \fillin[tuple]
            \item \lstinline|(i for i in range(3))| \hspace{1em} \fillin[générateur]
            \item \lstinline|'abc'| \hspace{1em} \fillin[chaine de caractères]
        \end{enumerate}
    \end{multicols}

    \question

    Quel est la sortie du code suivant \lstinline|"12314231423225".split("3")| ?

       \begin{multicols}{2}
    \begin{checkboxes}
        \choice \lstinline|['12314231421225']|
        \choice \lstinline|['1', '2', '4', '2', '1', '2', '25']|
        \CorrectChoice \lstinline|['12', '142', '142', '225']|
        \choice \lstinline|{1,2,3,4,2,3,1,4,2,3,2,2,5}|
        \choice Segmentation fault
    \end{checkboxes}
            \end{multicols}

    \question

    Je souhaite importer la fonction \lstinline|sqrt| sans polluer l'espace de nom courant de mon programme quelle est la bonne commande ?

       \begin{multicols}{2}
    \begin{checkboxes}
        \choice \lstinline{import sqrt}
        \choice \lstinline{from math import sqrt}
        \choice \lstinline{from sqrt import math}
        \choice \lstinline{using math.sqrt}
        \choice \lstinline{from math import *}
    \end{checkboxes}
         \end{multicols}

    \clearpage
    \question

    Python est un langage à typage dynamique. Qu'est-ce que cela signifie ?

    \begin{checkboxes}
        \choice Les variables doivent avoir un type spécifié avant utilisation
        \CorrectChoice Le type d'une variable est déterminé automatiquement au moment de l'exécution
        \choice Python n'alloue que des variables de type entier
        \choice Python utilise des types fixes comme dans C
    \end{checkboxes}

    \question

    Les modèles de données en Python définissent des comportements spécifiques pour les objets. Quelle méthode spéciale est typiquement appelée à l'initialisation d'une instance de classe ?

    \fillwithlines{1cm}

    \question

    Quelles sont les deux méthodes spéciales utilisées pour obtenir un itérateur et l'utiliser ?

    \begin{enumerate}
        \item \fillin[\lstinline|__iter__|]
        \item \fillin[\lstinline|__next__|]
    \end{enumerate}

    \question

    Quels sont les noms souvent utilisés pour définir 1. les actions d'un objet et 2. les variables d'état d'un objet ?

    \begin{enumerate}
        \item \fillin[méthodes]
        \item \fillin[attributs]
    \end{enumerate}

    \question

    Quelle est la meilleure définition de \lstinline|self| en Python ?

    \begin{checkboxes}
        \choice C'est un mot clé qui permet de définir une fonction
        \choice C'est un mot clé qui permet de définir une variable
        \choice C'est un mot clé utilisé dans les décorateurs
        \CorrectChoice C'est une convention pour référencer l'instance de la classe actuelle dans une méthode
    \end{checkboxes}

    \question

    Le mot clé \lstinline|next| est utilisé dans quel contexte en Python ?

    \begin{checkboxes}
        \choice Pour sauter une itération dans une boucle \lstinline|for|
        \CorrectChoice Pour obtenir le prochain élément d'un itérateur
        \choice Pour obtenir le prochain élément d'un dictionnaire
        \choice Pour arrêter une boucle \lstinline|for|
    \end{checkboxes}

    \question

    Quelle est la valeur de \lstinline|x| après l'exécution du code suivant ?

    \begin{lstlisting}
u = {1: 20, 3: 40, 5: 60}
x = len(u)
\end{lstlisting}

    \begin{multicols}{4}
        \begin{checkboxes}
            \choice 20
            \CorrectChoice 3
            \choice 4
            \choice 5
            \choice 6
            \choice 40
            \choice 60
        \end{checkboxes}
    \end{multicols}

    \question

    Que sera affiché sur la sortie standard ?

    \begin{lstlisting}
x = [1, 2, 3]
y = x
y.append(4)
print(x)
\end{lstlisting}

    \begin{multicols}{2}
        \begin{checkboxes}
            \choice \lstinline|[3, 2, 1]|
            \choice \lstinline|[1, 2, 3]|
            \CorrectChoice \lstinline|[1, 2, 3, 4]|
            \choice \lstinline|[1, 2, 3, 4, 4]|
            \choice \lstinline|[1, 2, 3, 1, 2, 3, 1, 2, 3, 1, 2, 3]|
            \choice \lstinline|None|
        \end{checkboxes}
    \end{multicols}

    \question

    Quelle est la valeur affichée sur la sortie standard ?

    \begin{lstlisting}
def foo(a, b=2, c=3):
    return a - b * c

result = foo(5, c=10)
print(result)
\end{lstlisting}

    \fillwithlines{1cm}

    \question

    Que voyez-vous sur la sortie standard ?

    \begin{lstlisting}
u = [1, 2, 3, 4, 5]
print([n**2 for n in u if n % 2 == 0])
    \end{lstlisting}

    \begin{multicols}{2}
        \begin{checkboxes}
            \choice \lstinline|[1, 4, 9, 16, 25]|
            \choice \lstinline|[1, 9, 25]|
            \choice \lstinline|[1, 2, 3, 4, 5]|
            \CorrectChoice \lstinline|[4, 16]|
        \end{checkboxes}
    \end{multicols}
    \question

    À quoi sert le module pip en Python ?

    \begin{checkboxes}
        \choice Compiler du code Python
        \choice Gérer les dépendances et automatiser la création d'environnements virtuels pour Python
        \CorrectChoice Installer des modules Python depuis le terminal
        \choice Optimiser les performances du code
        \choice Relier la sortie standard d'un programme à un autre

    \end{checkboxes}

    \question

    Quel est le type de \lstinline|**kwargs| dans la signature d'une fonction Python ?
    \begin{multicols}{2}
        \begin{checkboxes}
            \choice Une liste
            \CorrectChoice Un dictionnaire
            \choice Un set
            \choice Un tuple
        \end{checkboxes}
    \end{multicols}

    \question

    Quelle est la différence principale entre une liste et un tuple en Python ?

    \begin{checkboxes}
        \choice Une liste est immuable, un tuple est mutable
        \CorrectChoice Un tuple est immuable, une liste est mutable
        \choice Une liste peut contenir des types différents, un tuple ne peut contenir qu'un seul type
        \choice Un tuple peut contenir des types différents, une liste ne peut contenir qu'un seul type
    \end{checkboxes}

    \question

    Quel est l'intérêt principal d'utiliser des environnements virtuels en Python ?

    \begin{checkboxes}
        \choice Améliorer les performances d'exécution des scripts Python
        \CorrectChoice Isoler les dépendances de projets Python pour éviter les conflits entre versions de paquets
        \choice Permettre l'exécution de scripts Python sur des systèmes d'exploitation différents
        \choice Faciliter le déploiement de scripts Python sur des serveurs distants
    \end{checkboxes}
\end{questions}
\end{document}
