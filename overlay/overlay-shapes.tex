% overlay-shapes.tex
\documentclass[a4paper]{article}
\usepackage[margin=2cm]{geometry}
\usepackage[T1]{fontenc}
\usepackage{lmodern}
\usepackage{tikz}
\usetikzlibrary{shapes.geometric,calc}
\pagestyle{empty}

% --- Mode de sortie : traits vs remplissage ---
% Par défaut: contours uniquement (pdf1).
% Pour "remplissage uniquement", compile avec:
%   pdflatex "\def\FILL{}% overlay-shapes.tex
\documentclass[a4paper]{article}
\usepackage[margin=2cm]{geometry}
\usepackage[T1]{fontenc}
\usepackage{lmodern}
\usepackage{tikz}
\usetikzlibrary{shapes.geometric,calc}
\pagestyle{empty}

% --- Mode de sortie : traits vs remplissage ---
% Par défaut: contours uniquement (pdf1).
% Pour "remplissage uniquement", compile avec:
%   pdflatex "\def\FILL{}% overlay-shapes.tex
\documentclass[a4paper]{article}
\usepackage[margin=2cm]{geometry}
\usepackage[T1]{fontenc}
\usepackage{lmodern}
\usepackage{tikz}
\usetikzlibrary{shapes.geometric,calc}
\pagestyle{empty}

% --- Mode de sortie : traits vs remplissage ---
% Par défaut: contours uniquement (pdf1).
% Pour "remplissage uniquement", compile avec:
%   pdflatex "\def\FILL{}% overlay-shapes.tex
\documentclass[a4paper]{article}
\usepackage[margin=2cm]{geometry}
\usepackage[T1]{fontenc}
\usepackage{lmodern}
\usepackage{tikz}
\usetikzlibrary{shapes.geometric,calc}
\pagestyle{empty}

% --- Mode de sortie : traits vs remplissage ---
% Par défaut: contours uniquement (pdf1).
% Pour "remplissage uniquement", compile avec:
%   pdflatex "\def\FILL{}\input{overlay-shapes.tex}"
\newif\iffillonly
\ifdefined\FILL
  \fillonlytrue
\fi

% Tailles des formes
\def\Size{2.2cm}      % pdf1 : grand contour
\def\InnerSize{1.3cm} % pdf2 : plus petit, rempli

% Styles selon le mode
\iffillonly
  \tikzset{shapeStyle/.style={draw=none, fill=black}}
\else
  \tikzset{shapeStyle/.style={draw=black, fill=none, line width=1pt}}
\fi

% ---------- Numérotation ----------
\newcommand{\CenterNumber}[1]{%
  \node at (current page.center) {\fontsize{160}{0}\selectfont #1};
}
\newcommand{\EdgeNumbers}[1]{%
  \node at ($(current page.north)+(0,-8mm)$) {#1}; % haut
  \node at ($(current page.south)+(0, 8mm)$) {#1}; % bas
  \node at ($(current page.west) +( 8mm,0)$) {#1}; % gauche
  \node at ($(current page.east) +(-8mm,0)$) {#1}; % droite
}

% Commande pour dessiner une page
\newcommand{\DrawPage}[1]{%
  \clearpage
  \thispagestyle{empty}
  \begin{tikzpicture}[remember picture, overlay]
    % Positions (marges intérieures de 1.5 cm)
    \def\TL{([xshift=1.5cm,yshift=-1.5cm]current page.north west)}   % haut gauche
    \def\BR{([xshift=-1.5cm,yshift=1.5cm]current page.south east)}   % bas droite

    % ===== Formes =====
    % Impaires : cercle (TL) + triangle (BR)
    % Paires   : carré (TL)  + octogone (BR)
    \ifodd#1
      \iffillonly
        \node[circle, minimum size=\InnerSize, shapeStyle] at \TL {};
        \node[regular polygon, regular polygon sides=3, minimum size=\InnerSize, shapeStyle] at \BR {};
      \else
        \node[circle, minimum size=\Size, shapeStyle] at \TL {};
        \node[regular polygon, regular polygon sides=3, minimum size=\Size, shapeStyle] at \BR {};
      \fi
    \else
      \iffillonly
        \node[regular polygon, regular polygon sides=4, minimum size=\InnerSize, shapeStyle] at \TL {};
        \node[regular polygon, regular polygon sides=8, minimum size=\InnerSize, shapeStyle] at \BR {};
      \else
        \node[regular polygon, regular polygon sides=4, minimum size=\Size, shapeStyle] at \TL {};
        \node[regular polygon, regular polygon sides=8, minimum size=\Size, shapeStyle] at \BR {};
      \fi
    \fi

    % ===== Numéros =====
    \iffillonly
      \EdgeNumbers{#1}    % pdf2 : numéros sur les arêtes
    \else
      \CenterNumber{#1}   % pdf1 : gros numéro au centre
    \fi
  \end{tikzpicture}
  \vfill
}

\begin{document}
\DrawPage{1}
\DrawPage{2}
\DrawPage{3}
\DrawPage{4}
\end{document}
"
\newif\iffillonly
\ifdefined\FILL
  \fillonlytrue
\fi

% Tailles des formes
\def\Size{2.2cm}      % pdf1 : grand contour
\def\InnerSize{1.3cm} % pdf2 : plus petit, rempli

% Styles selon le mode
\iffillonly
  \tikzset{shapeStyle/.style={draw=none, fill=black}}
\else
  \tikzset{shapeStyle/.style={draw=black, fill=none, line width=1pt}}
\fi

% ---------- Numérotation ----------
\newcommand{\CenterNumber}[1]{%
  \node at (current page.center) {\fontsize{160}{0}\selectfont #1};
}
\newcommand{\EdgeNumbers}[1]{%
  \node at ($(current page.north)+(0,-8mm)$) {#1}; % haut
  \node at ($(current page.south)+(0, 8mm)$) {#1}; % bas
  \node at ($(current page.west) +( 8mm,0)$) {#1}; % gauche
  \node at ($(current page.east) +(-8mm,0)$) {#1}; % droite
}

% Commande pour dessiner une page
\newcommand{\DrawPage}[1]{%
  \clearpage
  \thispagestyle{empty}
  \begin{tikzpicture}[remember picture, overlay]
    % Positions (marges intérieures de 1.5 cm)
    \def\TL{([xshift=1.5cm,yshift=-1.5cm]current page.north west)}   % haut gauche
    \def\BR{([xshift=-1.5cm,yshift=1.5cm]current page.south east)}   % bas droite

    % ===== Formes =====
    % Impaires : cercle (TL) + triangle (BR)
    % Paires   : carré (TL)  + octogone (BR)
    \ifodd#1
      \iffillonly
        \node[circle, minimum size=\InnerSize, shapeStyle] at \TL {};
        \node[regular polygon, regular polygon sides=3, minimum size=\InnerSize, shapeStyle] at \BR {};
      \else
        \node[circle, minimum size=\Size, shapeStyle] at \TL {};
        \node[regular polygon, regular polygon sides=3, minimum size=\Size, shapeStyle] at \BR {};
      \fi
    \else
      \iffillonly
        \node[regular polygon, regular polygon sides=4, minimum size=\InnerSize, shapeStyle] at \TL {};
        \node[regular polygon, regular polygon sides=8, minimum size=\InnerSize, shapeStyle] at \BR {};
      \else
        \node[regular polygon, regular polygon sides=4, minimum size=\Size, shapeStyle] at \TL {};
        \node[regular polygon, regular polygon sides=8, minimum size=\Size, shapeStyle] at \BR {};
      \fi
    \fi

    % ===== Numéros =====
    \iffillonly
      \EdgeNumbers{#1}    % pdf2 : numéros sur les arêtes
    \else
      \CenterNumber{#1}   % pdf1 : gros numéro au centre
    \fi
  \end{tikzpicture}
  \vfill
}

\begin{document}
\DrawPage{1}
\DrawPage{2}
\DrawPage{3}
\DrawPage{4}
\end{document}
"
\newif\iffillonly
\ifdefined\FILL
  \fillonlytrue
\fi

% Tailles des formes
\def\Size{2.2cm}      % pdf1 : grand contour
\def\InnerSize{1.3cm} % pdf2 : plus petit, rempli

% Styles selon le mode
\iffillonly
  \tikzset{shapeStyle/.style={draw=none, fill=black}}
\else
  \tikzset{shapeStyle/.style={draw=black, fill=none, line width=1pt}}
\fi

% ---------- Numérotation ----------
\newcommand{\CenterNumber}[1]{%
  \node at (current page.center) {\fontsize{160}{0}\selectfont #1};
}
\newcommand{\EdgeNumbers}[1]{%
  \node at ($(current page.north)+(0,-8mm)$) {#1}; % haut
  \node at ($(current page.south)+(0, 8mm)$) {#1}; % bas
  \node at ($(current page.west) +( 8mm,0)$) {#1}; % gauche
  \node at ($(current page.east) +(-8mm,0)$) {#1}; % droite
}

% Commande pour dessiner une page
\newcommand{\DrawPage}[1]{%
  \clearpage
  \thispagestyle{empty}
  \begin{tikzpicture}[remember picture, overlay]
    % Positions (marges intérieures de 1.5 cm)
    \def\TL{([xshift=1.5cm,yshift=-1.5cm]current page.north west)}   % haut gauche
    \def\BR{([xshift=-1.5cm,yshift=1.5cm]current page.south east)}   % bas droite

    % ===== Formes =====
    % Impaires : cercle (TL) + triangle (BR)
    % Paires   : carré (TL)  + octogone (BR)
    \ifodd#1
      \iffillonly
        \node[circle, minimum size=\InnerSize, shapeStyle] at \TL {};
        \node[regular polygon, regular polygon sides=3, minimum size=\InnerSize, shapeStyle] at \BR {};
      \else
        \node[circle, minimum size=\Size, shapeStyle] at \TL {};
        \node[regular polygon, regular polygon sides=3, minimum size=\Size, shapeStyle] at \BR {};
      \fi
    \else
      \iffillonly
        \node[regular polygon, regular polygon sides=4, minimum size=\InnerSize, shapeStyle] at \TL {};
        \node[regular polygon, regular polygon sides=8, minimum size=\InnerSize, shapeStyle] at \BR {};
      \else
        \node[regular polygon, regular polygon sides=4, minimum size=\Size, shapeStyle] at \TL {};
        \node[regular polygon, regular polygon sides=8, minimum size=\Size, shapeStyle] at \BR {};
      \fi
    \fi

    % ===== Numéros =====
    \iffillonly
      \EdgeNumbers{#1}    % pdf2 : numéros sur les arêtes
    \else
      \CenterNumber{#1}   % pdf1 : gros numéro au centre
    \fi
  \end{tikzpicture}
  \vfill
}

\begin{document}
\DrawPage{1}
\DrawPage{2}
\DrawPage{3}
\DrawPage{4}
\end{document}
"
\newif\iffillonly
\ifdefined\FILL
  \fillonlytrue
\fi

% Tailles des formes
\def\Size{2.2cm}      % pdf1 : grand contour
\def\InnerSize{1.3cm} % pdf2 : plus petit, rempli

% Styles selon le mode
\iffillonly
  \tikzset{shapeStyle/.style={draw=none, fill=black}}
\else
  \tikzset{shapeStyle/.style={draw=black, fill=none, line width=1pt}}
\fi

% ---------- Numérotation ----------
\newcommand{\CenterNumber}[1]{%
  \node at (current page.center) {\fontsize{160}{0}\selectfont #1};
}
\newcommand{\EdgeNumbers}[1]{%
  \node at ($(current page.north)+(0,-8mm)$) {#1}; % haut
  \node at ($(current page.south)+(0, 8mm)$) {#1}; % bas
  \node at ($(current page.west) +( 8mm,0)$) {#1}; % gauche
  \node at ($(current page.east) +(-8mm,0)$) {#1}; % droite
}

% Commande pour dessiner une page
\newcommand{\DrawPage}[1]{%
  \clearpage
  \thispagestyle{empty}
  \begin{tikzpicture}[remember picture, overlay]
    % Positions (marges intérieures de 1.5 cm)
    \def\TL{([xshift=1.5cm,yshift=-1.5cm]current page.north west)}   % haut gauche
    \def\BR{([xshift=-1.5cm,yshift=1.5cm]current page.south east)}   % bas droite

    % ===== Formes =====
    % Impaires : cercle (TL) + triangle (BR)
    % Paires   : carré (TL)  + octogone (BR)
    \ifodd#1
      \iffillonly
        \node[circle, minimum size=\InnerSize, shapeStyle] at \TL {};
        \node[regular polygon, regular polygon sides=3, minimum size=\InnerSize, shapeStyle] at \BR {};
      \else
        \node[circle, minimum size=\Size, shapeStyle] at \TL {};
        \node[regular polygon, regular polygon sides=3, minimum size=\Size, shapeStyle] at \BR {};
      \fi
    \else
      \iffillonly
        \node[regular polygon, regular polygon sides=4, minimum size=\InnerSize, shapeStyle] at \TL {};
        \node[regular polygon, regular polygon sides=8, minimum size=\InnerSize, shapeStyle] at \BR {};
      \else
        \node[regular polygon, regular polygon sides=4, minimum size=\Size, shapeStyle] at \TL {};
        \node[regular polygon, regular polygon sides=8, minimum size=\Size, shapeStyle] at \BR {};
      \fi
    \fi

    % ===== Numéros =====
    \iffillonly
      \EdgeNumbers{#1}    % pdf2 : numéros sur les arêtes
    \else
      \CenterNumber{#1}   % pdf1 : gros numéro au centre
    \fi
  \end{tikzpicture}
  \vfill
}

\begin{document}
\DrawPage{1}
\DrawPage{2}
\DrawPage{3}
\DrawPage{4}
\end{document}
