\clearpage

\titledquestion{Programmation}[10]

Écrire un programme complet en C permettant de calculer la résistance équivalente d'un réseau de N résistances montées en parallèle, selon la formule suivante:

\[
R_{\mathrm{eq}} = \left(\frac{1}{R_1} + \frac{1}{R_2} + \frac{1}{R_3} + \cdots\right)^{-1}
\]

\begin{center}
\begin{circuitikz}
  \ctikzset{resistor = european}

  \draw (0,3) node[ocirc,label=left:$A$]{} -- (8,3); % rail haut (A)
  \draw (0,0) node[ocirc,label=left:$B$]{} -- (8,0); % rail bas  (B)

  \draw (2,3) to[R,l_=$R_1$,*-*] (2,0);
  \draw (4,3) to[R,l_=$R_2$,*-*] (4,0);
  \draw (6,3) to[R,l_=$R_3$,*-*] (6,0);
\end{circuitikz}
\end{center}

Les grandeurs sont passées au programme au moyen des arguments de la ligne de commande. Le programme exécuté avec 5 résistances recevra donc 6 arguments: le nom du programme suivi des 5 valeurs des résistances.

Le programme doit être \textbf{robuste}, c'est-à-dire détecter les entrées incorrectes et ne pas \emph{planter}. Toute valeur manquante, nulle ou négative constitue une entrée invalide.

Vous pouvez vous aider de \lstinline{sscanf} pour valider les données.

Voici un exemple de fonctionnement:

\begin{code}{bash}{}
$ ./parallel 120 220 330
54.6796
$ ./parallel 100 100 100 100 100
20.0
$ echo $?
0
$ ./parallel 47 -10 10
Error: Invalid resistance values
$ echo $?
1
\end{code}

\begin{solutionordottedlines}[\stretch{1}]
\begin{lstlisting}
#include <stdio.h>

int main(int argc, char *argv[]) {
    double inv_sum = 0.0;
    for (int i = 1; i < argc; ++i) {
        if (argv[i] == NULL) {
            printf("Paramètres invalides\n");
            return 1;
        }
        double r;
        if (sscanf(argv[i], "%lf", &r) != 1 || r <= 0) {
            printf("Paramètres invalides\n");
            return 1;
        }
        inv_sum += 1.0 / r;
    }

    double req = 1.0 / inv_sum;
    printf("%g\n", req);
}
\end{lstlisting}
\end{solutionordottedlines}

\ifprintanswers
\else
    \clearpage
    \fillwithdottedlines{\stretch{1}}
\fi
