\clearpage
\noaddpoints
\titledquestion{Entrées \& sorties}[\totalpoints]
\addpoints

\begin{parts}
    \part
    Pour les appels de fonction \lstinline{printf} suivants, indiquez l'affichage exact produit sur \emph{stdout} ou, en cas d'erreur, la nature de cette dernière. Représentez la sortie dans l'espace approprié, un caractère par case. Utilisez la notation des caractères d'échappement du langage C pour les caractères non imprimables (p.~ex. \lstinline{\n}). Terminez chaque sortie par une croix sur toute la case.

    Considérez les états des variables suivants:
    \begin{code}{c}{}
short s = -27;
unsigned char uc = 201;
char c = 'c'; // Valeur ASCII de 'c': 99
double x = 2.71828;
    \end{code}

    \vskip 2em
    \setlength{\gridsize}{7mm}
    \setlength{\gridlinewidth}{0.1pt}

    \begin{subparts}

        \subpart[1] \lstinline{printf("|0x%05x|\n", uc);}
        \begin{solutionorgrid}[1cm]
            \lstinline!|0x000c9|\n!
        \end{solutionorgrid}

        \subpart[1] \lstinline{printf("%+7.2f\n", s / 2.0);}
        \begin{solutionorgrid}[1cm]
            \lstinline! -13.50\n!
        \end{solutionorgrid}

        \subpart[1] \lstinline{printf("%c%c%c%c %hhd", c, c - 2, 'a' + 2, 97, c);}
        \begin{solutionorgrid}[1cm]
            \lstinline!caca 99!
        \end{solutionorgrid}

        \subpart[1] \lstinline{printf(">%-06.1f<", x);}
        \begin{solutionorgrid}[1cm]
            \lstinline!>0002.7<!
        \end{solutionorgrid}

        \subpart[1] \lstinline{printf("%-*.*f%d\n", 7, 3, 5.4321, 42);}
        \begin{solutionorgrid}[1cm]
            \lstinline!5.432  42!
        \end{solutionorgrid}

    \end{subparts}

    \questionanchorhere[-break]

    \newpage
    \part Entrée standard.

    Soient les déclarations suivantes:

    \begin{code}{c}{}
int r = 0, n = 0, m = 0;
double y = 0.0;
char ch = '0';
    \end{code}
    Pour les appels \lstinline{sscanf} qui suivent, indiquez:

    \begin{enumerate}
        \item la valeur des variables affectées;
        \item la valeur de retour \lstinline{r}.
    \end{enumerate}

    \vskip 2em
    \begin{subparts}

        \subpart[1] \lstinline{r = sscanf("  -15 0x1f", "%d %i", &n, &m);}
        \answerline[\lstinline{r = 2; n = -15; m = 31;}]

        \subpart[1] \lstinline{r = sscanf("42kg", "%2d%c", &n, &ch);}
        \answerline[\lstinline{r = 2; n = 42; ch = 'k';}]

        \subpart[1] \lstinline{r = sscanf("0.75,12", "%lf,%d", &y, &n);}
        \answerline[\lstinline{r = 2; y = 0.75; n = 12;}]

        \subpart[1] \lstinline{r = sscanf("abc", "%d", &n);}
        \answerline[\lstinline{r = 0;}]

        \subpart[1] \lstinline{r = sscanf("9.82", "%d %d", &n, &m);}
        \answerline[\lstinline{r = 1; n = 9;}]

    \end{subparts}

\questionanchorhere[-end]
\end{parts}
