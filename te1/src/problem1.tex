\titledquestion{Numération}[10]

\setlength\answerlinelength{5 cm}

Pour chacune des constantes littérales suivantes indiquez leur équivalent:

\begin{enumerate}
    \item binaire,
    \item octal,
    \item décimal signé,
    \item décimal non signé et
    \item hexadécimal.
\end{enumerate}

Considérez que les nombres sont stockés en mémoire sur \textbf{8 bits} et que la représentation signée utilise le \textbf{complément à deux}.
Complétez le tableau ci-dessous en remplissant les cases vides.

\vskip 2em

\renewcommand{\arraystretch}{2}

\begin{center}
    \begin{tabular}{ccccc}
        \textbf{bin}             & \textbf{octal}     & \textbf{int}        & \textbf{uint}     & \textbf{hex}       \\
        \midrule
        |0b01011010|             & \fillin[0132][2cm] & \fillin[+90][2cm]   & \fillin[90][2cm]  & \fillin[0x5a][2cm] \\
        \fillin[0b11000111][2cm] & \fillin[0307][2cm] & \fillin[-57][2cm]   & |199|             & \fillin[0xc7][2cm] \\
        \fillin[0b10010000][2cm] & \fillin[0220][2cm] & |-112|              & \fillin[144][2cm] & \fillin[0x90][2cm] \\
        \fillin[0b01111101][2cm] & |0175|             & \fillin[+125][2cm]  & \fillin[125][2cm] & \fillin[0x7d][2cm] \\
        \fillin[0b11111010][2cm] & \fillin[0372][2cm] & \fillin[-6][2cm]    & \fillin[250][2cm] & |0xfa|             \\

    \end{tabular}
\end{center}
\vspace{1cm}
\questionanchorhere[-end]
\begin{solution}
    Pour retrouver la valeur signée d'un nombre négatif, appliquez la règle du complément à deux :
    inversion des bits puis ajout de 1. Le passage en base octale s'obtient en regroupant les bits
    par paquets de trois à partir de la droite, et l'hexadécimal en paquets de quatre.
\end{solution}
