\documentclass[french,a4paper,addpoints,11pt]{exam}
\usepackage{fontspec}
\usepackage[french]{babel}
\usepackage{hexam}
\usepackage{bashful}
\usepackage{booktabs}
\usepackage{hyperref}

\makeatletter
\renewcommand{\questionlabel}{%
  \Hy@raisedlink{\hypertarget{Q\thequestion-anchor}{}}%
  \thequestion.
}
\renewcommand{\partlabel}{%
  \Hy@raisedlink{\hypertarget{part@\arabic{question}@\arabic{partno}-anchor}{}}%
  (\thepartno)
}
\renewcommand{\subpartlabel}{%
  \Hy@raisedlink{\hypertarget{subpart@\arabic{question}@\arabic{partno}@\arabic{subpart}-anchor}{}}%
  \thesubpart.
}
\newcommand{\questionanchorhere}[1][]{%
  \Hy@raisedlink{\hypertarget{Q\thequestion#1-anchor}{}}%
}
\makeatother

\lstMakeShortInline|
\pointsdroppedatright
\renewcommand{\checkboxeshook}{%
  \settowidth{\leftmargin}{W.}%
  \labelwidth\leftmargin\advance\labelwidth-\labelsep%
}
\renewcommand{\partshook}{\setlength{\itemsep}{1em}}

\setlength{\gridsize}{5mm}
\setlength{\gridlinewidth}{0.1pt}
\colorgrids
\definecolor{GridColor}{gray}{0.5}
\definecolor{FillWithDottedLinesColor}{gray}{0.5}

\thispagestyle{empty}

\title{Travail écrit Info1-TIN-B\\TE-01}

\author{Prof. Yves Chevallier}
\department{Département TIN}
\school{Haute École d'Ingénierie et de Gestion du Canton de Vaud}

\date{4 novembre 2025}

\begin{document}

\begin{coverpages}
    \maketitle
    \thispagestyle{empty}

    \begin{center}
        \vskip 2em
        \gradetable
    \end{center}

    \vspace{0.1in}

    \fbox{\fbox{\parbox{5.5in}{
                \textbf{Consignes :}
                \vskip 1em%
                \begin{itemize}
                    \item Écrire votre \textbf{nom} et votre \textbf{prénom} sur la première page.
                    \item Écrire \textbf{lisiblement}, au stylo ou au crayon à papier gras.
                    \item Répondre aux questions dans les espaces appropriés.
                    \item Relire toutes vos réponses avant de rendre votre travail.
                    \item Rendre toutes les feuilles de ce travail écrit.
                    \item Les réponses données sur les feuilles de brouillon ne sont \textbf{pas acceptées}.
                    \item Aucun moyen de communication autorisé.
                    \item Toutes les réponses concernent le langage C et son standard C20.
                \end{itemize}
            }}}

\end{coverpages}
\setcounter{page}{2}
\begin{questions}
    \titledquestion{Numération}[10]

\setlength\answerlinelength{5 cm}

Pour chacune des constantes littérales suivantes indiquez leur équivalent:

\begin{enumerate}
    \item binaire,
    \item octal,
    \item décimal signé,
    \item décimal non signé et
    \item hexadécimal.
\end{enumerate}

Considérez que les nombres sont stockés en mémoire sur \textbf{8 bits} et que la représentation signée utilise le \textbf{complément à deux}.
Complétez le tableau ci-dessous en remplissant les cases vides.

\vskip 2em

\renewcommand{\arraystretch}{2}

\begin{center}
    \begin{tabular}{ccccc}
        \textbf{bin}             & \textbf{octal}     & \textbf{int}        & \textbf{uint}     & \textbf{hex}       \\
        \midrule
        |0b01011010|             & \fillin[0132][2cm] & \fillin[+90][2cm]   & \fillin[90][2cm]  & \fillin[0x5a][2cm] \\
        \fillin[0b11000111][2cm] & \fillin[0307][2cm] & \fillin[-57][2cm]   & |199|             & \fillin[0xc7][2cm] \\
        \fillin[0b10010000][2cm] & \fillin[0220][2cm] & |-112|              & \fillin[144][2cm] & \fillin[0x90][2cm] \\
        \fillin[0b01111101][2cm] & |0175|             & \fillin[+125][2cm]  & \fillin[125][2cm] & \fillin[0x7d][2cm] \\
        \fillin[0b11111010][2cm] & \fillin[0372][2cm] & \fillin[-6][2cm]    & \fillin[250][2cm] & |0xfa|             \\

    \end{tabular}
\end{center}
\vspace{1cm}
\questionanchorhere[-end]
\begin{solution}
    Pour retrouver la valeur signée d'un nombre négatif, appliquez la règle du complément à deux :
    inversion des bits puis ajout de 1. Le passage en base octale s'obtient en regroupant les bits
    par paquets de trois à partir de la droite, et l'hexadécimal en paquets de quatre.
\end{solution}

    \clearpage
\noaddpoints
\titledquestion{Syntaxe}[\totalpoints]
\addpoints

\setlength\answerlinelength{8cm}

\begin{parts}

    \part
    Pour chacun des identificateurs suivants, indiquez s'ils sont corrects selon le standard C, et si non, proposez un nom compatible en C.
    \begin{subparts}

        \subpart[1] |résumé|
        \answerline[\textbf{Incorrect} en raison de l'accent. \\ Proposition \lstinline{summary}.]

        \subpart[1] |_AMountain_|
        \answerline[\textbf{Correct}.]

        \subpart[1] \texttt{while}
        \answerline[\textbf{Incorrect} : mot clé du langage. \\ Proposition \lstinline{case_value}.]

        \subpart[1] |std99|
        \answerline[\textbf{Correct}.]

        \subpart[1] |data-stream|
        \answerline[\textbf{Incorrect} : le tiret n'est pas permis. \\ Proposition \lstinline{data_stream}.]

\end{subparts}

\questionanchorhere[-break]

    \part

    Pour chaque constante littérale suivante, indiquez si elle est \textbf{correcte} et, si applicable, précisez le \textbf{type} associé.

    \begin{subparts}
        \subpart[1] |42uL|
        \answerline[\textbf{Correct}, \lstinline{unsigned long int}.]

        \subpart[1] |0758|
        \answerline[\textbf{Incorrect} : la base octale ne permet pas le chiffre \lstinline{8}.]

        \subpart[1] |.128f|
        \answerline[\textbf{Correct}, \lstinline{float}.]

        \subpart[1] |'\x41'|
        \answerline[\textbf{Correct}, \lstinline{char}.]

        \subpart[1] |0b00001010|
        \answerline[\textbf{Correct}, mais à partir de C23, \lstinline{int}.]
    \end{subparts}

\questionanchorhere[-end]
\end{parts}

    \clearpage
\noaddpoints
\titledquestion{Entrées \& sorties}[\totalpoints]
\addpoints

\begin{parts}
    \part
    Pour les appels de fonction \lstinline{printf} suivants, indiquez l'affichage exact produit sur \emph{stdout} ou, en cas d'erreur, la nature de cette dernière. Représentez la sortie dans l'espace approprié, un caractère par case. Utilisez la notation des caractères d'échappement du langage C pour les caractères non imprimables (p.~ex. \lstinline{\n}). Terminez chaque sortie par une croix sur toute la case.

    Considérez les états des variables suivants:
    \begin{code}{c}{}
short s = -27;
unsigned char uc = 201;
char c = 'c'; // Valeur ASCII de 'c': 99
double x = 2.71828;
    \end{code}

    \vskip 2em
    \setlength{\gridsize}{7mm}
    \setlength{\gridlinewidth}{0.1pt}

    \begin{subparts}

        \subpart[1] \lstinline{printf("|0x%05x|\n", uc);}
        \begin{solutionorgrid}[1cm]
            \lstinline!|0x000c9|\n!
        \end{solutionorgrid}

        \subpart[1] \lstinline{printf("%+7.2f\n", s / 2.0);}
        \begin{solutionorgrid}[1cm]
            \lstinline! -13.50\n!
        \end{solutionorgrid}

        \subpart[1] \lstinline{printf("%c%c%c%c %hhd", c, c - 2, 'a' + 2, 97, c);}
        \begin{solutionorgrid}[1cm]
            \lstinline!caca 99!
        \end{solutionorgrid}

        \subpart[1] \lstinline{printf(">%-06.1f<", x);}
        \begin{solutionorgrid}[1cm]
            \lstinline!>0002.7<!
        \end{solutionorgrid}

        \subpart[1] \lstinline{printf("%-*.*f%d\n", 7, 3, 5.4321, 42);}
        \begin{solutionorgrid}[1cm]
            \lstinline!5.432  42!
        \end{solutionorgrid}

    \end{subparts}

    \questionanchorhere[-break]

    \newpage
    \part Entrée standard.

    Soient les déclarations suivantes:

    \begin{code}{c}{}
int r = 0, n = 0, m = 0;
double y = 0.0;
char ch = '0';
    \end{code}
    Pour les appels \lstinline{sscanf} qui suivent, indiquez:

    \begin{enumerate}
        \item la valeur des variables affectées;
        \item la valeur de retour \lstinline{r}.
    \end{enumerate}

    \vskip 2em
    \begin{subparts}

        \subpart[1] \lstinline{r = sscanf("  -15 0x1f", "%d %i", &n, &m);}
        \answerline[\lstinline{r = 2; n = -15; m = 31;}]

        \subpart[1] \lstinline{r = sscanf("42kg", "%2d%c", &n, &ch);}
        \answerline[\lstinline{r = 2; n = 42; ch = 'k';}]

        \subpart[1] \lstinline{r = sscanf("0.75,12", "%lf,%d", &y, &n);}
        \answerline[\lstinline{r = 2; y = 0.75; n = 12;}]

        \subpart[1] \lstinline{r = sscanf("abc", "%d", &n);}
        \answerline[\lstinline{r = 0;}]

        \subpart[1] \lstinline{r = sscanf("9.82", "%d %d", &n, &m);}
        \answerline[\lstinline{r = 1; n = 9;}]

    \end{subparts}

\questionanchorhere[-end]
\end{parts}

    \clearpage
\noaddpoints
\titledquestion{Structures de contrôle}[\totalpoints]
\addpoints

Donnez les valeurs affichées sur \emph{stdout} lors des séquences suivantes:

\begin{parts}

    \part[2] \begin{code}{c}{}
int s = 1;
while (s < 20) {
    printf("%d ", s);
    s <<= 1;
}
    \end{code}
    \answerline[1 2 4 8 16 ]

    \part[2] \begin{code}{c}{}
for (int i = 5; i > 0; --i) {
    if (i % 2 == 0) continue;
    printf("%d", i);
}
    \end{code}
    \answerline[531]

    \part[2] \begin{code}{c}{}
int total = 0;
for (int i = 1; i <= 3; ++i) {
    for (int j = i; j <= 3; ++j) {
        total += j;
    }
}
printf("%d", total);
    \end{code}
    \answerline[14]

    \part[2] \begin{code}{c}{}
for (int i = 0; i < 3; ++i) {
    for (int j = 0; j < 3; ++j) {
        if (i == j) break;
        printf("%d%d ", i, j);
    }
}
    \end{code}
    \answerline[10 20 21 ]

    \part[2] \begin{code}{c}{}
int i = 1, t = 0;
do {
    t += i;
    printf("%d;", t);
    i += 2;
} while (t < 15);
    \end{code}
    \answerline[1;4;9;16;]

\end{parts}

    \clearpage

\titledquestion{Programmation}[10]

Écrire un programme complet en C permettant de calculer la résistance équivalente d'un réseau de N résistances montées en parallèle, selon la formule suivante:

\[
R_{\mathrm{eq}} = \left(\frac{1}{R_1} + \frac{1}{R_2} + \frac{1}{R_3} + \cdots\right)^{-1}
\]

\begin{center}
\begin{circuitikz}
  \ctikzset{resistor = european}

  \draw (0,3) node[ocirc,label=left:$A$]{} -- (8,3); % rail haut (A)
  \draw (0,0) node[ocirc,label=left:$B$]{} -- (8,0); % rail bas  (B)

  \draw (2,3) to[R,l_=$R_1$,*-*] (2,0);
  \draw (4,3) to[R,l_=$R_2$,*-*] (4,0);
  \draw (6,3) to[R,l_=$R_3$,*-*] (6,0);
\end{circuitikz}
\end{center}

Les grandeurs sont passées au programme au moyen des arguments de la ligne de commande. Le programme exécuté avec 5 résistances recevra donc 6 arguments: le nom du programme suivi des 5 valeurs des résistances.

Le programme doit être \textbf{robuste}, c'est-à-dire détecter les entrées incorrectes et ne pas \emph{planter}. Toute valeur manquante, nulle ou négative constitue une entrée invalide.

Vous pouvez vous aider de \lstinline{sscanf} pour valider les données.

Voici un exemple de fonctionnement:

\begin{code}{bash}{}
$ ./parallel 120 220 330
54.6796
$ ./parallel 100 100 100 100 100
20.0
$ echo $?
0
$ ./parallel 47 -10 10
Error: Invalid resistance values
$ echo $?
1
\end{code}

\begin{solutionordottedlines}[\stretch{1}]
\begin{lstlisting}
#include <stdio.h>

int main(int argc, char *argv[]) {
    double inv_sum = 0.0;
    for (int i = 1; i < argc; ++i) {
        if (argv[i] == NULL) {
            printf("Paramètres invalides\n");
            return 1;
        }
        double r;
        if (sscanf(argv[i], "%lf", &r) != 1 || r <= 0) {
            printf("Paramètres invalides\n");
            return 1;
        }
        inv_sum += 1.0 / r;
    }

    double req = 1.0 / inv_sum;
    printf("%g\n", req);
}
\end{lstlisting}
\end{solutionordottedlines}

\ifprintanswers
\else
    \clearpage
    \fillwithdottedlines{\stretch{1}}
\fi

\end{questions}
\end{document}
