\clearpage
\noaddpoints
\titledquestion{Syntaxe}[\totalpoints]
\addpoints

\setlength\answerlinelength{8cm}

\begin{parts}

    \part
    Pour chacun des identificateurs suivants, indiquez s'ils sont corrects selon le standard C, et si non, proposez un nom compatible en C.
    \begin{subparts}

        \subpart[1] |résumé|
        \answerline[\textbf{Incorrect} en raison de l'accent. \\ Proposition \lstinline{summary}.]

        \subpart[1] |_AMountain_|
        \answerline[\textbf{Correct}.]

        \subpart[1] \texttt{while}
        \answerline[\textbf{Incorrect} : mot clé du langage. \\ Proposition \lstinline{case_value}.]

        \subpart[1] |std99|
        \answerline[\textbf{Correct}.]

        \subpart[1] |data-stream|
        \answerline[\textbf{Incorrect} : le tiret n'est pas permis. \\ Proposition \lstinline{data_stream}.]

\end{subparts}

\questionanchorhere[-break]

    \part

    Pour chaque constante littérale suivante, indiquez si elle est \textbf{correcte} et, si applicable, précisez le \textbf{type} associé.

    \begin{subparts}
        \subpart[1] |42uL|
        \answerline[\textbf{Correct}, \lstinline{unsigned long int}.]

        \subpart[1] |0758|
        \answerline[\textbf{Incorrect} : la base octale ne permet pas le chiffre \lstinline{8}.]

        \subpart[1] |.128f|
        \answerline[\textbf{Correct}, \lstinline{float}.]

        \subpart[1] |'\x41'|
        \answerline[\textbf{Correct}, \lstinline{char}.]

        \subpart[1] |0b00001010|
        \answerline[\textbf{Correct}, mais à partir de C23, \lstinline{int}.]
    \end{subparts}

\questionanchorhere[-end]
\end{parts}
