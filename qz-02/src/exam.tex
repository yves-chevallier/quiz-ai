\documentclass[french,a4paper,addpoints,11pt]{exam}
\usepackage[T1]{fontenc}
\usepackage[french]{babel}
\usepackage{fontspec}
\usepackage{hexam}
\usepackage{bashful}
\usepackage{booktabs}
\usepackage{hyperref}     % formulaires / widgets PDF
\usepackage{tikz}         % positionnement overlay
\usepackage{everyshi}

\usepackage{tikz}
\usetikzlibrary{calc}
\usepackage{xcolor}
\usepackage{etoolbox}
\usepackage{zref-savepos}




\usetikzlibrary{calc}
\hypersetup{pdfborder={0 0 0}, colorlinks=false} % pas de bordures visibles

\setlength{\itemsep}{0pt}
\setlength{\parskip}{0pt}
\setlength{\topsep}{0pt}

\lstMakeShortInline|
\pointsdroppedatright
\renewcommand{\checkboxeshook}{%
  \settowidth{\leftmargin}{W.}
  \setlength{\topsep}{1em}
  \labelwidth\leftmargin\advance\labelwidth-\labelsep%
}
\renewcommand{\partshook}{\setlength{\itemsep}{1em}}

\setlength{\gridsize}{5mm}
\setlength{\gridlinewidth}{0.1pt}
\colorgrids
\definecolor{GridColor}{gray}{0.5}
\definecolor{FillWithDottedLinesColor}{gray}{0.5}

\thispagestyle{empty}

\title{Quiz Python\\ QZ-02}

\author{Prof. Yves Chevallier}
\department{Département TIN}
\school{Haute École d'Ingénierie et de Gestion du Canton de Vaud}

\date{4 novembre 2025}

\makeatletter
\renewcommand{\questionlabel}{%
  \Hy@raisedlink{\hypertarget{Q\thequestion-anchor}{}}%
  \thequestion.
}
\makeatother

\begin{document}

\begin{coverpages}
    \maketitle
    \thispagestyle{empty}

    \vspace{0.1in}

    \fbox{\fbox{\parbox{5.5in}{
                \textbf{Consignes :}
                \vskip 1em%
                \begin{itemize}
                    \item Écrire votre nom et votre prénom sur la première page.
                    \item Écrire lisiblement, au stylo ou au crayon à papier gras.
                    \item Répondre aux questions dans les espaces appropriés.
                    \item Relire toutes vos réponses avant de rendre votre travail.
                    \item Rendre toutes les feuilles de ce quiz.
                    \item Aucun moyen de communication autorisé.
                    \item Toutes les réponses concernent le langage Python 3.10.
                    \item Cochez les questions à choix multiple avec une croix \textbf{X}.
                \end{itemize}
            }}}

\end{coverpages}
\setcounter{page}{2}
\begin{questions}

    \question Quel énoncé décrit le mieux le concept d'une blockchain ?

    \begin{checkboxes}
        \CorrectChoice Chaque bloc contient un \emph{hash} du bloc précédent, ce qui les relie en chaîne.
        \choice Chaque bloc à un lien vers ses parents et ses déscendants.
        \choice Une blockchain est dossier spécialisé dans le stockage de fichiers.
        \choice La blockchain est un protocole de communication entre serveurs web.
    \end{checkboxes}


    \question Qu'appelle-t-on une transformation non bijective ?

    \begin{checkboxes}
        \choice Une transformation qui est à la fois injective et surjective
        \CorrectChoice Une transformation qui n'est pas inversible car son espace d'arrivée est de dimension inférieure à son espace de départ
        \choice Une transformation linéaire qui conserve les distances
        \choice La fonction sinus ou cosinus par exemple
    \end{checkboxes}

    \question Quel est l'intérêt principal d'une fonction de hachage cryptographique comme SHA-256 ?

    \begin{multicols}{2}
        \begin{checkboxes}
            \choice Chiffrer un message de bout en bout
            \choice Compresser un fichier sans perte
            \CorrectChoice Générer une signature de taille fixe
            \choice Convertir un nombre en binaire
        \end{checkboxes}
    \end{multicols}

    \question Quelle affirmation décrit correctement la différence entre \lstinline|git merge| et \lstinline|git rebase| ?

    \begin{checkboxes}
        \choice \lstinline|git rebase| supprime l'historique, \lstinline|git merge| le conserve
        \choice Les deux commandes produisent exactement les mêmes commits
        \choice \lstinline|git merge| n'est disponible que sur GitHub
        \CorrectChoice \lstinline|git merge| crée un commit de fusion alors que \lstinline|git rebase| réécrit l'historique pour appliquer vos commits après une autre branche
    \end{checkboxes}

    \question Dans quel cas la création d'une branche Git est-elle conseillée ?
        \begin{checkboxes}
            \choice Pour sauvegarder automatiquement les fichiers temporaires
            \choice Pour exécuter un script Python
            \CorrectChoice Lorsqu'on démarre une nouvelle fonctionnalité ou un correctif indépendant
            \choice Pour supprimer rapidement des commits
        \end{checkboxes}

    \question Quel est le principe de la preuve de travail (\emph{proof-of-work}) utilisée par Bitcoin ?

    \begin{checkboxes}
        \choice Les utilisateurs doivent prouver leur identité auprès d'une autorité centrale
        \CorrectChoice Les mineurs doivent calculer un \emph{hash} commençant par un certain nombre de zéros pour valider un bloc
        \choice Les transactions sont validées par un vote majoritaire non pondéré
        \choice Le protocole duplique chaque bloc sur plusieurs chaînes concurrentes
    \end{checkboxes}

    \clearpage
    \question Quelle sortie affiche le code suivant ?

\begin{lstlisting}
letters = ['a', 'b', 'c']
numbers = [1, 2, 3]
print(list(zip(letters, numbers)))
\end{lstlisting}

    \begin{multicols}{2}
        \begin{checkboxes}
            \choice \lstinline|['a', 'b', 'c', 1, 2, 3]|
            \choice \lstinline|{'a': 1, 'b': 2, 'c': 3}|
            \CorrectChoice \lstinline|[('a', 1), ('b', 2), ('c', 3)]|
            \choice \lstinline|[('a', 'b', 'c'), (1, 2, 3)]|
        \end{checkboxes}
    \end{multicols}

    \question À propos de \lstinline|**kwargs| et de \lstinline|kwargs.get|, quelle affirmation est correcte ?

    \begin{checkboxes}
        \CorrectChoice \lstinline|**kwargs| regroupe les arguments nommés dans un dictionnaire et \lstinline|kwargs.get('clé', défaut)| retourne une valeur sans lever d'exception si la clé est absente
        \choice \lstinline|**kwargs| crée automatiquement des variables globales
        \choice \lstinline|**kwargs| est utilisé pour les arguments positionnels
        \choice \lstinline|**kwargs| ne peut être utilisé qu'avec des nombres
    \end{checkboxes}

    \question Un dossier peut être considéré comme un module importable en Python à condition de...

    \begin{checkboxes}
        \CorrectChoice Contenir un fichier \lstinline|__init__.py| même vide
        \choice Contenir un fichier \lstinline|requirements.txt|
        \choice Porter le même nom que le projet Git
        \choice Contenir des fichiers \lstinline|.py|
    \end{checkboxes}

    \question Comment réalise-t-on un import relatif à partir d'un module ?

    \begin{checkboxes}
        \choice En préfixant le chemin par \lstinline|import://|
        \choice En ajoutant le chemin complet du disque dans l'instruction \lstinline|import|
        \choice En déclarant une variable globale appelée \lstinline|__relative__|
        \CorrectChoice En utilisant \lstinline|from .sous_module import nom| ou \lstinline|from ..package import nom|
    \end{checkboxes}

    \question À quoi sert la fonction \lstinline|enumerate| dans une boucle \lstinline|for| ?

    \begin{checkboxes}
        \CorrectChoice Retourner à la fois l'indice et la valeur lors de l'itération
        \choice Trier automatiquement la liste avant l'itération
        \choice Convertir une liste en dictionnaire
        \choice Arrêter la boucle après trois éléments
    \end{checkboxes}

    \question Quel est l'avantage principal du mot clé \lstinline|with| lors de l'ouverture d'un fichier en Python ?

    \begin{checkboxes}
        \CorrectChoice Il garantit la fermeture du fichier même en cas d'exception
        \choice Il chiffre automatiquement le contenu du fichier
        \choice Il supprime le fichier après lecture
        \choice Il copie le fichier dans un dossier temporaire
    \end{checkboxes}

    \question Si on souhaite surcharger l'addition pour l'objet \lstinline|foo| dans le cas \lstinline|42 + foo| ?

    \begin{checkboxes}
        \choice Il faut surcharger la méthode \lstinline|__add__| dans la classe de \lstinline|foo|
        \CorrectChoice Il faut surcharger la méthode \lstinline|__radd__| dans la classe de \lstinline|foo|
        \choice Il faut définir une fonction \lstinline|add(foo)| en dehors de la classe
        \choice Il faut utiliser la fonction intégrée \lstinline|sum()|
    \end{checkboxes}

    \question Quelle est la sortie du code suivant ?

\begin{lstlisting}
nums = [1, 2, 3, 4]
print(nums[1:-1])
\end{lstlisting}

    \begin{multicols}{2}
        \begin{checkboxes}
            \CorrectChoice \lstinline|[2, 3]|
            \choice \lstinline|[1, 2, 3, 4]|
            \choice \lstinline|[1, 4]|
            \choice \lstinline|[3]|
        \end{checkboxes}
    \end{multicols}

    \question Quel est l'effet d'un bloc \lstinline|try ... except| bien écrit ?

    \begin{checkboxes}
        \CorrectChoice Intercepter et traiter une exception pour éviter l'arrêt brutal du programme
        \choice Accélérer l'exécution d'un programme Python
        \choice Empêcher complètement qu'une exception se produise
        \choice Convertir automatiquement une chaîne de caractères en nombre
    \end{checkboxes}

    \question Quelle est la particularité principale d'un objet \lstinline|set| en Python ?

    \begin{checkboxes}
        \CorrectChoice Il ne contient pas de doublons et ne garantit pas l'ordre
        \choice Il conserve l'ordre d'insertion et autorise les doublons
        \choice Il ne peut contenir que des chaînes de caractères
        \choice Il sert uniquement à associer des clés et des valeurs
    \end{checkboxes}

    \question Quelle est la sortie de l'expression \lstinline|[n * n for n in range(3)]| ?

    \begin{multicols}{2}
        \begin{checkboxes}
            \choice \lstinline|[1, 4, 9]|
            \choice \lstinline|[0, 1, 2, 3]|
            \CorrectChoice \lstinline|[0, 1, 4]|
            \choice \lstinline|[2, 4]|
        \end{checkboxes}
    \end{multicols}

    \question Que produit l'instruction \lstinline|list(range(5))| ?

    \begin{multicols}{2}
        \begin{checkboxes}
            \choice \lstinline|[1, 2, 3, 4, 5]|
            \choice \lstinline|[0, 5]|
            \choice \lstinline|[5, 4, 3, 2, 1]|
            \CorrectChoice \lstinline|[0, 1, 2, 3, 4]|
        \end{checkboxes}
    \end{multicols}

    \question À quoi sert le test \lstinline|if __name__ == "__main__":| dans un module Python ?

    \begin{checkboxes}
        \CorrectChoice À exécuter un bloc de code seulement lorsque le module est lancé comme script principal
        \choice À empêcher toute importation du module
        \choice À définir le point d'entrée obligatoire d'une bibliothèque
        \choice À vérifier la version de Python installée
    \end{checkboxes}

    \question Quelle affirmation décrit correctement \lstinline|*args| dans la définition d'une fonction Python ?

    \begin{checkboxes}
        \choice Il transforme automatiquement les arguments en dictionnaire
        \CorrectChoice Il capture les arguments positionnels supplémentaires dans un tuple
        \choice Il impose que la fonction n'accepte qu'un seul argument
        \choice Il convertit les paramètres en chaînes de caractères
    \end{checkboxes}

\end{questions}
\end{document}
